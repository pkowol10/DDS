\documentclass{beamer}
\usetheme[parttitle=leftfooter]{AGH}
\usepackage[utf8]{inputenc}
\usepackage{polski}
\usepackage[polish]{babel}

\usepackage{csvsimple}
\usepackage{tikz}
\usepackage{pgfplots}
\usepackage{pgfplotstable}

\usepackage[european, american inductor, american currents]{circuitikz}
\pgfplotsset{compat=1.18}

\usepackage[natbib=true, backend=biber, style=numeric, sorting=none]{biblatex}
\addbibresource{../Dokumentacja/references.bib}

\usepackage{subcaption}

\setbeamertemplate{bibliography item}{\insertbiblabel}

% \setbeamertemplate{headline}{
%     \leavevmode%
%     \hbox{%
%         \begin{beamercolorbox}[wd=0.5\paperwidth,ht=2.5ex,dp=1ex, left]{author in head/foot}%
        
%         \end{beamercolorbox}%
%         \begin{beamercolorbox}[wd=0.5\paperwidth,ht=2.5ex,dp=1ex, left]{date in head/foot}%
        
%         \end{beamercolorbox}%
%     }
% }

\setbeamertemplate{footline}
{
    \leavevmode%
    \hbox{%
    \begin{beamercolorbox}[wd=.33\paperwidth,ht=2.25ex,dp=1ex,left]{author in head/foot}%
        \usebeamerfont{author in head/foot}\insertshortauthor
    \end{beamercolorbox}%
    \begin{beamercolorbox}[wd=.34\paperwidth,ht=2.25ex,dp=1ex,center]{title in head/foot}%
        \usebeamerfont{author in head/foot}\insertshortinstitute
    \end{beamercolorbox}%
    \begin{beamercolorbox}[wd=.33\paperwidth,ht=2.25ex,dp=1ex,right]{date in head/foot}%
        \usebeamerfont{date in head/foot}\insertshortdate{}\hspace*{3em}
        \insertframenumber{} / \inserttotalframenumber\hspace*{1ex}
    \end{beamercolorbox}
    }%
    \vskip0pt%
}

\renewcommand{\bibfont}{\small}

\setbeamerfont{title}{size={\fontsize{13}{13}}}
% \fontsize{0.8}{0.8}
\title{Szybki generator funkcyjny sterujący drabinką R-2R}
% \fontsize{1}{1}
\author{Piotr Kowol, Mateusz Kulak, Piotr Łętowski\\ \footnotesize Opiekun projektu:Ernest Jamro}
\institute{Akademia Górniczo-Hutnicza w Krakowie}
\date[\today]

\begin{document}
    \maketitle

    \begin{frame}{Plan Prezentacji}
        \tableofcontents
    \end{frame}

    \section{Założenia projektowe}
    \begin{frame}{Założenia projektowe}
        \begin{block}{}
            \begin{itemize}
                \item Odtwarzanie sygnałów wykorzystując bezpośrednią syntezę cyfrową
                \item Wykorzystanie dedykowanych serializerów w celu uzyskania wyższej częstotliwości odtwarzania
                \item Możliwość załadowania próbek sygnału do pamięci
            \end{itemize}
        \end{block}
    \end{frame}

    \section{Schemat blokowy układu}
    \begin{frame}{Schemat blokowy układu}
        \begin{figure}
            \centering
            \scalebox{0.5}{\begin{subfigure}{\textwidth}
    \hspace{-5cm}
    \begin{tikzpicture}
        \draw
            (0, 0) node[draw, rectangle, minimum width = 2.5cm, minimum height = 2.5cm, label = {[align = center]center:Akumulator \\fazy}](P_acc){}
            (4, 0) node[draw, rectangle, minimum width = 2.5cm, minimum height = 2.5cm, label = {[align = center]center:Pamięć \\próbek}](LUT){}
            (8, 0) node[draw, rectangle, minimum width = 2.5cm, minimum height = 2.5cm, label = {[align = center]center:Serializer}](SER){}
            (12, 0) node[draw, rectangle, minimum width = 2.5cm, minimum height = 2.5cm, label = {[align = center]center:8-bit \\przetwornik \\C/A}](DAC){}
            (16, 0) node[draw, rectangle, minimum width = 2.5cm, minimum height = 2.5cm, label = {[align = center]center:FDP}](FDP){}
            (4, -4) node[draw, rectangle, minimum width = 2.5cm, minimum height = 2.5cm, label = {[align = center]center:Mikroprocesor}](uP){}
            (4, -8) node[draw, rectangle, minimum width = 2.5cm, minimum height = 2.5cm, label = {[align = center]center:Interfejs \\UART}](UART){}
            (-0.25, -8) node[draw, rectangle, minimum width = 2cm, minimum height = 2cm, label = {[align = center]center:Generacja \\zegara}](CLK){}

            (P_acc) ++ (1.25, 0) -- ++ (1.5, 0) node[inputarrow]{}
            (P_acc) ++ (1.9, -0.1) -- ++ (0.2, 0.2) ++ (-0.1, -0.1) node[above]{addr} node[below]{N} 
            
            (P_acc) ++ (-0.5, -2.5) -- ++ (0, 1.25) node[inputarrow, rotate = 90]{}
            (P_acc) ++ (-0.5, -1.9) node[left]{LS\_CLK}

            (LUT) ++ (1.25, 0) -- ++ (1.5, 0) node[inputarrow]{}
            (LUT) ++ (1.9, -0.1) -- ++ (0.2, 0.2) ++ (-0.1, -0.1) node[above]{data} node[below]{K} 

            (LUT) ++ (-2, -2.5) |- ++ (0.75, 2) node[inputarrow]{}
            (LUT) ++ (-2, -1.9) node[left]{LS\_CLK}

            (SER) ++ (1.25, 0) -- ++ (1.5, 0) node[inputarrow]{}
            (SER) ++ (1.9, -0.1) -- ++ (0.2, 0.2) ++ (-0.1, -0.1) node[above]{sample} node[below]{8} 

            (SER) ++ (-0.5, -2.5) -- ++ (0, 1.25) node[inputarrow, rotate = 90]{}
            (SER) ++ (-0.5, -1.9) node[left]{LS\_CLK}

            (SER) ++ (0.5, -2.5) -- ++ (0, 1.25) node[inputarrow, rotate = 90]{}
            (SER) ++ (0.5, -1.9) node[right]{HS\_CLK}

            (DAC) ++ (1.25, 0) -- ++ (1.5, 0) node[inputarrow]{}
            (DAC) ++ (2, 0) node[above]{signal} 

            (DAC) ++ (0.5, -2.5) -- ++ (0, 1.25) node[inputarrow, rotate = 90]{}
            (DAC) ++ (0.5, -1.9) node[right]{HS\_CLK}

            (FDP) ++ (1.25, 0) -- ++ (1, 0) node[inputarrow]{}
            (FDP) ++ (2, 0) node[above]{Out} 

            (uP) ++ (-1.25, 0) -| ++ (-2.75, 2.75) node[inputarrow, rotate = 90]{}
            (uP) ++ (-2.25, 0.1) -- ++ (-0.2, -0.2) ++ (0.1, 0.1) node[above]{step} node[below]{M} 

            (uP) ++ (-1, 1.25) -- ++ (0, 1.5) node[inputarrow, rotate = 90]{}
            (uP) ++ (-1.1, 2.15) -- ++ (0.2, 0.2) node[left]{addr} node[right]{N}

            (uP) ++ (0, 1.25) -- ++ (0, 1.5) node[inputarrow, rotate = 90]{}
            (uP) ++ (-0.1, 1.65) -- ++ (0.2, 0.2) node[left]{data} node[right]{K}

            (uP) ++ (-0.5, -1.25) -- ++ (0, -1.5) node[inputarrow, rotate = -90]{}
            (uP) ++ (-0.5, -2) node[left]{TxD}

            (uP) ++ (-2.5, -1) -- ++ (1.25, 0) node[inputarrow]{}
            (uP) ++ (-1.9, -1) node[above]{LS\_CLK}

            (uP) ++ (-1.25, 1) -| ++ (-2.25, 1.75) node[inputarrow, rotate = 90]{}
            (uP) ++ (-2.75, 1) node[above]{ctrl}

            (uP) ++ (1, 1.25) -- ++ (0, 1.5) node[inputarrow, rotate = 90]{}
            (uP) ++ (1, 2) node[right]{ctrl}

            (uP) ++ (1.25, 0) -| ++ (2.75, 2.75) node[inputarrow, rotate = 90]{}
            (uP) ++ (2.25, 0) node[above]{ctrl}

            (UART) ++ (0.5, 1.25) -- ++ (0, 1.5) node[inputarrow, rotate = 90]{}
            (UART) ++ (0.5, 2) node[left]{RxD}

            (CLK) ++ (1, 0.5) -- ++ (1.25, 0) node[inputarrow]{}
            (CLK) ++ (1.75, 0.5) node[above]{HS\_CLK}

            (CLK) ++ (1, -0.5) -- ++ (1.25, 0) node[inputarrow]{}
            (CLK) ++ (1.75, -0.5) node[above]{LS\_CLK}
        ;
    \end{tikzpicture}
\end{subfigure}}
            \caption{Schemat blokowy generatora.}
            \label{sch:DDS}
        \end{figure}
    \end{frame}

    \subsection{Akumulator fazy}
    \begin{frame}{Akumulator fazy}
        \begin{block}{}
            \begin{itemize}
                \item Składa się z rejestru i sumatora
                \item Z każdym taktem LS\_CLK zwiększa zawartość rejestru o zadany krok fazowy
                \item Adresuje pamięć próbek
            \end{itemize}
        \end{block}
    \end{frame}

    \subsection{Pamięć próbek}
    \begin{frame}{Pamięć próbek}
        \begin{block}{}
            \begin{itemize}
                \item Pamięć dwu-portowa - jeden port tylko do zapisu, a drugi tylko do odczytu
                \item Przechowuje wartości generowanego przebiegu
                \item Przekazuje próbki do serializera z każdym taktem LS\_CLK
                \item Zrealizowana w postaci pamięci BRAM
            \end{itemize}
        \end{block}
    \end{frame}

    \subsection{Serializer}
    \begin{frame}{Serializer}
        \begin{block}{}
            \begin{itemize}
                \item Wykorzystuje 8 dedykowanych układów OSERDESE2 w konfiguracji serializerów 8 do 1
                \item Przekazuje próbki do przetwornika C/A przy każdym zboczu HS\_CLK (DDR)
            \end{itemize}
        \end{block}
    \end{frame}

    \subsection{PRzetwornik C/A}
    \begin{frame}{przetwornik C/A}
        \begin{block}{}
            \begin{itemize}
                \item Drabinka R-2R na zewnątrz układu FPGA
                \item Zamienia próbki na schodkowy sygnał analogowy
            \end{itemize}
        \end{block}
    \end{frame}
    
    \subsection{Filtr dolnoprzepustowy}
    \begin{frame}{Filtr dolnoprzepustowy}
        \begin{block}{}
            \begin{itemize}
                \item Usuwa składowe wysokoczęstotliwościowe z generowanego sygnału
                \item Pozwala uzyskać przebieg analogowy z sygnału schodkowego
            \end{itemize}
        \end{block}
    \end{frame}

    \subsection{Mikroprocesor}
    \begin{frame}{Mikroprocesor}
        \begin{block}{}
            \begin{itemize}
                \item Konfiguruje pracę poszczególnych bloków
                \item Odbiera wartości próbek i dane konfiguracyjne z komputera za pośrednictwem protokołu UART
                \item Przekazuje próbki do pamięci
                \item Zrealizowany w postaci procesora ZYNQ w układzie FPGA
                \item Realizaja UARTu poprzez program napisany w C
            \end{itemize}
        \end{block}
    \end{frame}

    \subsection{Generacja zegara}
    \begin{frame}{Generacja zegara}
        \begin{block}{}
            \begin{itemize}
                \item Szybki zegar HS\_CLK do taktowania układów OSERDESE2 o częstotliwości 500 MHz (1 Gb/s)
                \item Wolny zegar LS\_CLK do taktowania pozostałej logiki oraz procesora ZYNQ o częstotliwości $\frac{1}{4} f_{HS\_CLK} = 125\ MHz$
                \item Wykorzystuje dedykowane bufory z programowalnym dzielnikiem częstotliwości BUFR do generowania LS\_CLK
            \end{itemize}
        \end{block}
    \end{frame}

    \begin{frame}
        \vspace*{2cm}
        \centering
        {\Huge Dziękujemy za uwagę.}
        \vspace*{\fill}
        \begin{flushright}
            Piotr Kowol, \\
            Mateusz Kulak, \\
            Piotr Łętowski
        \end{flushright}
    \end{frame}

    \begin{frame}[t, allowframebreaks]{Bibliografia}
        \nocite{*}
        \printbibliography
    \end{frame}

\end{document}