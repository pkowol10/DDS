\begin{subfigure}{\textwidth}
\hspace{0cm}
    \begin{tikzpicture}[node distance = 2cm]

    \node (start) [startstop] {START};
    \node (odczyt) [io, below of = start] {Odczyt rejestrów kontrolnych};
    \node (czyCE) [decision, below of = odczyt, yshift = -1cm] {Czy CE = 1};
    \node (czyLD) [decision, below of = czyCE, yshift = -2.5cm] {Czy załadować dane};
    \node (RD) [io, left of = czyLD, xshift = -4cm] {Odczyt danych};
    \node (LD) [process, below of = RD] {Załadowanie danych};
    \node (Rkrok) [io, below of = czyLD, yshift = -1.5cm] {Odczyt kroku fazowego};
    \node (Lkrok) [process, below of = Rkrok] {Ustawienie kroku fazowego};
    \node (czyGEN) [decision, below of = Rkrok, yshift = -3.5cm] {Czy rozpocząć generację};
    \node (GEN) [process, below of = czyGEN, yshift = -2cm] {Generowanie sygnału};
    \node (stop) [startstop, below of = GEN] {STOP};

    \draw [arrow] (start) -- (odczyt);
    \draw [arrow] (odczyt) -- (czyCE);
    \draw [arrow] (czyCE) --  node[anchor = east]{tak} (czyLD);
    \draw [arrow] (czyCE) -- node[anchor = south]{nie} ++ (4.5, 0) |- (stop);
    \draw [arrow] (czyLD) -- node[anchor = east]{nie} (Rkrok);
    \draw [arrow] (czyLD) -- node[anchor = south]{tak} (RD);
    \draw [arrow] (RD) -- (LD);
    \draw [arrow] (LD) -- ++ (-3.5, 0) |- (odczyt);
    \draw [arrow] (Rkrok) -- (Lkrok);
    \draw [arrow] (Lkrok) -- (czyGEN);
    \draw [arrow] (czyGEN) -- node[anchor = south]{nie} ++ (-9.5, 0) |- (odczyt);
    \draw [arrow] (czyGEN) -- node[anchor = east]{tak} (GEN);
    \draw [arrow] (GEN)  -- ++ (-9.5, 0) |- (odczyt);

    \end{tikzpicture}
\end{subfigure}