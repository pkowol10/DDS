\section{Wstęp}
    Generatory funkcyjne pozwalają generować sygnały elektryczne o zadanych parametrach. 
    Jednym ze sposobów na generowanie sygnału jest bezpośrednia synteza cyfrowa - DDS (\textit{ang.} 
    Direct Digital Synthesis). Jeden ze sposobów realizacji bezpośredniej syntezy cyfrowej 
    wykorzystuje pamięć próbek działającą na zasadzie pamięci LUT (\textit{ang.} Look-Up Table) oraz 
    akumulator fazy. W pamięci znajdują się kolejne wartości cyfrowe sygnału, a akumulator fazy z 
    każdym taktem zegara adresuje odpowiednie komórki pamięci. Następnie wartości próbek są przekazywana 
    do przetwornika cyfrowo-analogowego, który przetwarza sygnał cyfrowy na postać analogową. W takich 
    układach czesto wykorzystuje się nadrzędny procesor, który dostarcza sygnały sterujące. 
    Dzięki temu układy DDS mogą być łatwo konfigurowane poprzez przeprogramowanie pamięci próbek 
    lub jednorazowo programowane przez producenta do odtwarzania konkretnego typu sygnałów np. scalone 
    układy DDS sygnałów sinusoidalnych. W ramach projektu zrealizowano układ bezpośredniej syntezy 
    cyfrowej z programowalną pamięcią próbek, zaprojektowany w języku Verilog. 